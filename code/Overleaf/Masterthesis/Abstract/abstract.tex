\begin{abstract}
The COVID-19 pandemic showed how important data can be in order to make the best decisions to reduce and control an outbreak. Most of this data is obtained by using computer models to simulate how the disease spreads through a population. This gives researches the ability to imitate lots of different scenarios and predict what countermeasures are most effective. This thesis aims to improve such a model, called Stride, by increasing the performance and its ease of use. To optimise the model by reducing the simulation runtimes, the code was examined as well as the data that it uses. Performance tests led to discovering a major optimisation by passing a shared pointer by reference instead of by value, resulting in almost half the time needed to run some simulations, which has been immediately used in the field upon discovery. Furthermore, multiple algorithms were presented and examined which all resulted in faster runtimes, and even more than quadrupling the speed of simulations compared the original model from the start. These algorithms were verified to generate correct results, for which a new verification tool was introduced. A domain specific language was presented, called EpiQL, that can be used to express how a simulation should run by writing declarative rules, without the need for extensive knowledge of the model. The process of writing a compiler for this language, which is written in Rust, was laid out as well as the future steps to complete this work in progress. 
\end{abstract}