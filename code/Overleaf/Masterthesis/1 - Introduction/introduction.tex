\chapter{Introduction}
When a new infectious disease (re-)emerges, it is crucial to know how it can be controlled and combated in order to save lives. The knowledge of these diseases is not something that is gained overnight. Extensive research, which consists of gathering and processing data over the course of years or even decades, is needed to completely understand the ins and outs of a disease. The data that has to be gathered and the methods that will be used for the processing are different depending on the disease. Today's technology offers a lot of possibilities for the collection of data with among other things e-questionnaires \cite{e-questionnaire} and IoT-based (internet of things) data collection systems \cite{iot_data_collection}. Likewise, there are lots of different processing options on the gathered data depending on what has to be achieved, with for example machine learning which can be used in cancer prognoses and predictions \cite{ml_cancer}. In general, diseases can be grouped in two different types: infectious and non-infectious. Next to finding a cure for either type, it is also of interest to know how infectious diseases spread and behave under certain circumstances. This information is necessary to get a better understanding of diseases or make predictions on how they will spread. For this type of research, computer models have been created to simulate the diseases' behaviours and how they spread through populations. They play a major role in disease outbreaks, such as the COVID-19 pandemic, by providing essential information and insights to governments and the World Health Organization (WHO) in their policy decision making \cite{modelling_importance}. Since time is of the essence when these policy decisions have to be made, the speed at which these models generate information is of great importance. This thesis aims to improve such a model, namely Stride, an individual-based simulator for the transmission of infectious diseases \cite{stride}. The first goal is to analyse the Stride code to increase its performance, thus fulfilling the need for fast information retrieval. When policy decisions are being considered, a multitude of scenarios are simulated on the models which may require them to run in different ways. Therefore, the other goal of this thesis is to increase the flexibility and extensibility of Stride by creating a domain specific language (DSL). The intent of this DSL is to make Stride's functionalities easier and more transparent to use on a more abstract level.
\\\\
To start off this thesis, the key concepts of infectious diseases, how models work, and how these models can be used on those diseases is explained in Chapter \ref{chapter:modelling_of_infectious_diseases}. Then, once a general grasp is obtained on this material, the focus shifts to Stride. Chapter \ref{chapter:stride} examines the design of Stride along with the data that is used. In Chapter \ref{chapter:sampling} the major optimisation of this thesis, which is through a sampling approach, is explained and evaluated. The DSL previously talked about and the process of creating it is explained in Chapter \ref{chapter:dsl}.
